Biomis: Biometric-Based Blockchain for Identity-Bound Cryptocurrency

Author: Huara Kazama
Version: 1.0
Date: August 2025

Abstract

Biomis is a blockchain platform linking cryptocurrency ownership to a user’s biometric identity. Unlike traditional systems relying on seed phrases or private keys, Biomis authenticates transactions via biometric verification, using zero-knowledge proofs (ZKPs) for privacy-preserving validation. This approach mitigates theft, fraud, and key-loss risks while enabling regulated, identity-bound applications.

1. Introduction

Traditional blockchain systems rely on knowledge-based security. Loss or theft of private keys results in permanent fund loss. Biomis eliminates this risk by binding control to biometric data (fingerprint, iris, facial recognition).

Problem 1: Seed phrases are vulnerable.

Problem 2: Key theft can lead to fraudulent transactions.

Problem 3: Lack of identity verification complicates regulated use cases.

Biomis solves these by integrating biometrics and ZKP with a Rust-based blockchain and smart contract ecosystem.

2. Problem Statement

Key challenges Biomis addresses:

Identity-bound control: Only the owner’s biometrics authorize transactions.

Secure multi-device access: Recovery is biometric-based, no seed phrases required.

Fraud reduction: Stolen keys alone cannot authorize actions.

Regulatory compliance: Identity-bound verification enables governance, voting, and audits.

3. Biomis Architecture

Components:

Blockchain Node

Rust-based for safety and speed.

Consensus: PoS + PoBT.

Ledger: RocksDB for high throughput.

Biometric Module

Fuzzy cryptography generates cryptographic keys from biometric input.

ZKP for private transaction verification.

Optional homomorphic encryption for secure processing.

Wallet Client

Scans biometrics, derives keys, signs transactions.

Multi-device recovery.

Smart Contracts

Rust-native for fast execution.

Optional Solidity contracts for token and PoBT logic.

API / Middleware

Exposes JSON-RPC endpoints.

Handles serialization, ZKP verification, and transaction submission.

Architecture Flow Diagram
+-----------------+
|  User Device    |
| (Biometric Scan)|
+--------+--------+
         |
         v
+-----------------+
| Biometric Module|
| Fuzzy Crypto +  |
| ZKP Verification|
+--------+--------+
         |
         v
+-----------------+
|      Wallet     |
| Transaction     |
| Signing         |
+--------+--------+
         |
         v
+-----------------+
| Blockchain Node |
| PoS + PoBT      |
| RocksDB Ledger  |
+--------+--------+
         |
         v
+-----------------+
| Smart Contracts |
| Rust & Solidity |
+-----------------+

4. Biometric Security

Fuzzy Cryptography: Produces consistent keys despite minor biometric variations.

Zero-Knowledge Proofs: Verifies identity without exposing raw biometric data.

Multi-device Recovery: Users can register devices; recovery is biometric-verified.

5. Why Biomis Matters

Eliminates seed phrase dependency: Reduces permanent account loss risk.

Prevents identity theft: Stolen keys alone do not allow control.

Enhances usability: Biometric authentication is faster and intuitive.

Supports privacy: ZKPs enable secure verification.

Bridges real-world identity with blockchain: Useful for regulated apps, voting, and digital certification.

6. Potential Applications

Secure payments via biometric verification.

Digital identity verification for banking, healthcare, and government.

NFT ownership tied to verified identity.

DAO governance with identity-bound voting to prevent Sybil attacks.

7. Technical Stack
Layer	Technology
Blockchain Node	Rust, RocksDB, PoS + PoBT
Smart Contracts	Rust-native, Solidity (EVM-compatible)
Biometric & ZKP Layer	Fuzzy Crypto, Zero-Knowledge Proofs, Homomorphic Encryption
Wallet Client	Rust, Multi-device Recovery, Signing
API / Middleware	Rust, Axum/Tokio, JSON-RPC Endpoints

8. References

Nakamoto, S. Bitcoin: A Peer-to-Peer Electronic Cash System. 2008. https://bitcoin.org/bitcoin.pdf

Bonneau, J., et al. Research Perspectives and Challenges for Bitcoin and Cryptocurrencies. IEEE Symposium on Security and Privacy, 2015.

Juels, A., Sudan, M. Fuzzy Extractors: How to Generate Strong Keys from Biometrics and Other Noisy Data. SIAM Journal on Computing, 2006.

Goldwasser, S., Micali, S., Rackoff, C. The Knowledge Complexity of Interactive Proof Systems. SIAM Journal on Computing, 1989.

Rosenberg, M., et al. Zero-Knowledge Proofs in Blockchain Applications. Journal of Cryptographic Engineering, 2020.

Garay, J., Kiayias, A., Leonardos, N. The Bitcoin Backbone Protocol: Analysis and Applications. EUROCRYPT 2015.

9. Conclusion

Biomis provides a biometric-first blockchain ecosystem that mitigates fraud, eliminates reliance on seed phrases, and ensures privacy-preserving identity verification. By combining Rust-based performance, ZKPs, and identity-bound smart contracts, Biomis offers a practical, secure, and user-friendly alternative to conventional cryptocurrency systems.

                                             Redefining trust through human identity